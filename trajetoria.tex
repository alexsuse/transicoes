\documentclass{article}
    % General document formatting
    \usepackage[margin=0.7in]{geometry}
    \usepackage[parfill]{parskip}
    \usepackage[utf8]{inputenc}
    
    % Related to math
    \usepackage{amsmath,amssymb,amsfonts,amsthm}

\begin{document}

\title{Mecanica relacional}
\maketitle

O campo da fisica moderna sempre se coloca como um corpo de conhecimento solido,
que fala de fatos precisos, explica o comportamento da materia, liquida gasosa
ou solida. A materia (ou energia) existe como presenca constante e ininterrupta
nos modelos, sempre dada e real. Essa limitacao nos modos de existir que a fisica
se permite estudar certamente nao e absoluta. Eu diria ainda que quando
novas formas de existir invadem a bem-guardada \emph{natureza}, esta se
desdobra cada vez em uma \emph{natureza} outra.

Podemos tomar de exemplo diversos momentos, como por exemplo o aparecimento da
mecanica quantica no comeco dos 1900, onde uma forma de existencia que
desobedecia as leis conhecidas passou ao longo dos anos de abstracao absurda
para um membro pleno do panteao de objetos existentes. Outro exemplo e a
introducao da nocao de campo que temos hoje, de uma propriedade inerente do
espaco, se propagando sem um meio. James Maxwell desenvolveu suas equacoes
para um campo se propagando em um eter, ainda nao era possivel \emph{existir}
uma onda que nao tenha um meio de propagacao na \emph{natureza}.

Pois claramente se tornou possivel para um sistema quantico existir como
\emph{natureza}. A ciencia moderna cada vez mais toma a fisica
quantica como equivalente a \emph{natureza}. Dado que quimica e quantica
aplicada, biologia e quimica aplicada e assim por diante, se conclui que o nosso
modo de existencia e so o quantico agregado em tamanhos muito grandes. Ou seja,
existiriamos da mesma forma como existem eletrons ou proteinas ou bacterias. A
existencia e presenca constante invariante nao so com o tempo, mas tambem entre
escalas de todo tamanho. Existir na \emph{natureza} da ciencia moderna exige
tanta invarianca e presenca que os humanos nao alcancam.

Se, como a ciencia moderna, aceitarmos que a existencia de um sistema fisico e
invariante e sempre a mesma, logo cabe entender mais de cada "existencia" que o
sistema existe, procurando formas de estudar o sistema cujas existencia seja
estavel, ou seja, autoestados, orbitas estacionarios, energias
fundamentais, todos caracterizando algum arranjo da materia que se mantem
presente ao longo do tempo. Essa certamente nao e a unica forma de se pensar
sobre a materia e
energia, como discutem Isabelle Stengers e Ilya Prigogine em \emph{a nova
alianca}. Nao so essa nao e a unica como a visao do mundo como mecanica nascida
com newton exclui o humano, dado que nem somos estacionarios nem cabemos na
\emph{natureza} da ciencia. Stengers e Prigogine mencionam a obra de Whitehead,
que procurou estabelecer uma filosofia da natureza a partir de acontecimentos e
das relacoes entre acontecimentos. Dizer que um objeto existe como quadro seria
dizer que nas relacoes e acontecimentos entre o objeto e outrem o objeto se
comporta como um quadro.

Note que acontecimentos e relacoes nao definem necessariamente um estado.
Seguindo a sugestao, vou considerar um mesmo problema de duas formas diferentes,
primeiro sem qualquer nocao de estado (exceto a implicita nas interacoes) e
entao olhando para os estados possivelmente ocupados pelo sistema


\section{Mecanica sem estados}

Se nao temos estados, podemos falar pelo menos de acontecimentos e relacoes
entre acontecimentos. Digamos que nosso sistema interage com o universo somente
atraves de duas interacoes, ou acontecimentos. Cada interacao e caracterizada
por um sinal e um tempo em que a interacao ocorre e absorve ou emite uma
quantidade $\epsilon$ de energia.i.

%seguindo a maneira de integrais de trajetoria, vou considerar um sistema
%caracterizado por duas possiveis interacoes $I^+, I^-$ que vamos atraves das
%propriedades das interacoes mostrar que tem de se alternar.


Tomemos entao duas interacoes,
caracterizadas por uma dada magnitude ou variacao de energia ($\epsilon$) e uma
direcao ($+,-$). 
Uma interacao que ocorre no tempo $t$ sera denotada por $I^+(t)$ ou $I^-(t)$.
Queremos garantir que duas interacoes de mesmo sinal sempre obedecem $I^+(t)
I^+(s) = I^+(t), t<s$, ja que depois de absorver energia no instante $t$ nao
permitimos outra interacao do mesmo tipo. Alem disso essa propriedade e
necessaria para que haja "continuidade" em t,s.\\

Uma duracao $D(t,s)$ representa um intervalo de tempo no qual a variacao de
energia e 0. Digamos ainda que $D^+(t,s)I^+(r) = D^+(t,s)$, a duracao
durante a qual a variacao foi nula que so pode ser seguida por ua interacao de
sinal negativo e vice versa.\\

$$
D^-(0,T) = D^-(0,t_1) \prod_{i=1}^{2k} (-1)^i I^{(-1)^i}(t_i) D^-(t_{2k}, T)
$$

Note que as duracoes $D^+(t,s)$ representam um intervalo de tempo onde 
$H(s)-H(t) = 0$ e apos o qual segue necessariamente uma interacao do tipo
$I^-(r), r > s$, uma vez que

$$
D^+(0,t)I^+(s)I^-(r) = D^+(0,t) I^-(r).
$$
Uma trajetoria para o nosso sistema pode entao ser caracterizada somente pelos
tempos nos quais interacoes acontecem, pela sequencia de interacoes peritidas
que sempre se alternam entre os dois operadores.

$$
\mathcal{T}(0, t_1, \ldots, t_{2*k}, T) = D^-(0,t_1) \prod_{i=1}^k I^+(t_{2*i})
I^-(t_{2*i-1})
$$

Definiremos entao uma media sobre todos os estados equivalentes, da seguinte
forma:
$$
\sum_{k=0}^\infty \int d\mu(\mathcal{T}_k = \{t_1, \ldots, t_{2k}\}) \mathcal{F}[\mathcal{T}_k]
$$
$\mu$ deve ser normalizavel e podemos alem disso exigir que a energia media do
sistema seja dada por uma constante $E$.

A energia do sistema e constante a menos dos instantes ${t_i}$ onde ocorrem
interacoes, onde a variacao e $+-\epsilon$ alternadamente.
Cabe escrever entao:
$$\partial_t\mathcal{H}(\mathcal{T}_k) = \epsilon \sum_{i=1}^2k (-1)^i \delta(t-t_i)
$$
Tomando $\mathcal{H}(0) = -\frac{\epsilon}{2}$ teremos entao:

$$
\mathcal{H}(t, {t_1,\ldots,t_{2k}}) = \epsilon \sum_{i=1}^{2k} (-1)^i \Theta(t-t_i) = \epsilon \sum_{i=1}^k (\Theta(t-t_{2*i})-\Theta(t-t_{2*i-1}))
$$
Ate agora estavamos sempre considerando que os tempos de cada interacao eram ordenados, mas podemos agora notar que

$$
\Theta(t-a)-\Theta(t-b) = \Theta((t-a)(b-t)), portanto \mathcal{H} = \epsilon \sum_{i=1}^k \Theta((t-t_{2*i})(t_{2*i-1}-t))
$$
Note que trocar um par de $t_k,t_j$ nunca muda a somatoria:
$$
\Theta((t-t_i)(a-t)) + \Theta((t-t_j)(b-t)) = \Theta(t-t_i)-\Theta(a-t)+\Theta(t-t_j)\Theta(b-t) = \Theta(t-t_j)-\Theta(a-t)+\Theta(t-t_j)-\Theta(b-t) = \Theta((t-t_j)(a-t)) + \Theta((t-t_i)(b-t)) 
$$
portanto nao precisamos mais nos preocupar com a ordenacao dos tempos de interacao. Isso nos levara a contar cada trajetoria com $k$ pares de interacoes $k!$ vezes se integrarmos sobre todos os tempos para tirar a media, portanto teriamos:

$$
E(\mathcal{H}) = \sum_k=0^\infty \int \prod_i=1^k dt_i \sum_{i=1}^k \Theta((t-t_{2*i})(t_{2*i-1}-t))
$$
Podemos integrar cada par de instantes separadamente:
$$
\int_0^T dt_1 \int_0^T dt_2 \Theta((t-t_{2})(t_1-t))= \int dt_2 t_2\Theta(t-t_2) - \int_0^T dt_1 t_1\Theta(t-t_1) = (T-t)^2/2 + (t-T)^2/2 = (T-t)^2
$$
Obtemos entao:
$$
E=\sum_k=0^\infty \frac{\epsilon}{k!} \int_0^t (T-t)^{2k} \mu_i = \sum_{k=0}^\infty \frac{\mu_k \epsilon T^{2k+1} }{k!(2k+1)}
$$
usando um principio de maxima entropia sobre $\mu$, teremos:
$$
L = -\sum_i \mu_i log(\mu_i) + \lambda_1(E-\sum_i \mu_i \mathcal{H}) + \lambda_2 (1-\sum_i \mu_i), \partial_\mu L = 0, \partial_\lambda L = 0
$$
Obtemos entao 
$$
\frac{\partial L}{\partial \lambda_1} = E - E(\mu) 
$$
$$
\frac{\partial L}{\partial \lambda_2} = 1- \sum_i \mu_i
$$
$$
\frac{\partial L}{\partial \mu_i} = -log(\mu_i) - 1 - \lambda_1 \frac{\epsilon T^{2i+1}}{(2i+1)i!} = 0
$$

E a solucao sera:
$$
\mu_i \propto exp(-\lambda_1 \epsilon \frac{T^{2*i+1}}{(2i+1)i!})
$$

Se agora computarmos a media sobre todas as 

respectivamente. Se as unicas interacoes permitidas ao nosso sistema forem essas
duas, notemos que Ambas interacoes tem de ser idempotentes. Caso $I^+(t) I^+(s)
\ne I^+(t)$
teriamos uma nova interacao diferente das dadas, ela poderia ser reescrita como
$I^+(t)I^-(t_1)I^+(t_2)I^+(s) = D(0,t_2)I^+_(t_2)I^+(s)$. Tomando $t_2 = 0,
D(0,0)I^+(0)I^+(s) = \mathcal{I}(s)$, 

\end{document}
